
\documentclass{hogent-article}
\bibliography{./bibliografie}
\usepackage{lipsum}
\usepackage{courier} %% Sets font for listing as Courier.
\usepackage{listings, xcolor}

\definecolor{javared}{rgb}{0.6,0,0} % for strings
\definecolor{javagreen}{rgb}{0.25,0.5,0.35} % comments
\definecolor{javapurple}{rgb}{0.5,0,0.35} % keywords
\definecolor{javadocblue}{rgb}{0.25,0.35,0.75} % javadoc

\lstdefinestyle{codeStyle}{language=Java,
    tabsize = 2, %% set tab space width
    showstringspaces = false, %% prevent space marking in strings, string is defined as the text that is generally printed directly to the console
    numbers = none, %% display line numbers on the left
    commentstyle = \color{green}, %% set comment color
    keywordstyle = \color{blue}, %% set keyword color
    stringstyle = \color{red}, %% set string color
    rulecolor = \color{black}, %% set frame color to avoid being affected by text color
    basicstyle = \small \ttfamily , %% set listing font and size
    breaklines = true, %% enable line breaking
    numberstyle = \tiny,
    frame = trBL, 
    firstnumber = last, 
    escapeinside={(*@}{@*)}}



\definecolor{codegreen}{rgb}{0,0.6,0}
\definecolor{codegray}{rgb}{0.5,0.5,0.5}
\definecolor{codepurple}{rgb}{0.58,0,0.82}
\definecolor{backcolour}{rgb}{0.95,0.95,0.92}


\lstdefinestyle{commentStyle}{
    backgroundcolor=\color{backcolour},   
    commentstyle=\color{codegreen},
    keywordstyle=\color{magenta},
    numberstyle=\tiny\color{codegray},
    stringstyle=\color{codepurple},
    basicstyle=\ttfamily\footnotesize,
    breakatwhitespace=false,         
    breaklines=true,                 
    captionpos=b,                    
    keepspaces=true,                 
    numbers=none,                    
    numbersep=5pt,                  
    showspaces=false,                
    showstringspaces=false,
    showtabs=false,                  
    tabsize=2
}



%------------------------------------------------------------------------------
% Metadata over het voorstel
%------------------------------------------------------------------------------

%---------- Titel & auteur ----------------------------------------------------

% TODO: geef werktitel van je eigen voorstel op
\PaperTitle{Coursework Distributed Databases}
\PaperType % Type document

% TODO: vul je eigen naam in als auteur, geef ook je emailadres mee!
\Authors{Strijk Philippe, Wout Boeykens, Hannes Roegiers\textsuperscript{1}} % Authors
\CoPromotor{\textsuperscript{2} (Van Vreckem Bert)}
\affiliation{\textbf{Contact:}
  \textsuperscript{1} \href{mailto:wout.boeykens@student.hogent.be}{wout.boeykens@student.hogent.be};
  \textsuperscript{2}
  \href{mailto:hannes.roegiers@student.hogent.be}{hannes.roegiers@student.hogent.be};
  \textsuperscript{3} \href{mailto:philippe.strijk@student.hogent.be}{philippe.strijk@student.hogent.be}

}

%---------- Abstract ----------------------------------------------------------

\Abstract{This task discusses 3 Kaggle competitions}


\Keywords{Apache --- Spark --- Maven --- Big Data --- Machine Learning --- Classification --- Regression --- Kaggle --- Distributed Databases} % Keywords
\newcommand{\keywordname}{Sleutelwoorden} % Defines the keywords heading name

%---------- Titel, inhoud -----------------------------------------------------

\begin{document}

\flushbottom % Makes all text pages the same height
\maketitle % Print the title and abstract box
\tableofcontents % Print the contents section
\thispagestyle{empty} % Removes page numbering from the first page

%------------------------------------------------------------------------------
% Hoofdtekst
%------------------------------------------------------------------------------

% De hoofdtekst van het voorstel zit in een apart bestand, zodat het makkelijk
% kan opgenomen worden in de bijlagen van de bachelorproef zelf.
%---------- Inleiding ---------------------------------------------------------

\section{Introduction} % The \section*{} command stops section numbering
\label{sec:introduction}

At the start of this task, our knowledge about Java Spark and the implementation of Machine Learning methods was quite limited, if not inexistant. Translating code from easy-to-use python scikit-learn to Java Apache Spark libraries proved to be a challenge that would send us down many knowledge-filled rabbit holes. The unique character of this assignment was evidently clear from the get-go.
This coursework discusses 3 Kaggle competitions, namely the Quora Insincere Question Classification, <Wout Subject>, <Hannes Subject>. 


%---------- Stand van zaken ---------------------------------------------------

\section{Data Preprocessing}
\label{sec:preprocessing}



\section{Model Selection}
\subsection{Logistic Regression}
\subsection{Random Forest Classifier}

%---------- Verwachte conclusies ----------------------------------------------
\section{Conclusions}
\label{sec:conclusions}

  



%------------------------------------------------------------------------------
% Referentielijst
%------------------------------------------------------------------------------


\phantomsection

\printbibliography[heading=bibintoc]

\end{document}
